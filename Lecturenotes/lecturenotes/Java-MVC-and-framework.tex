\chapter{MVC和框架初步}
\label{chp:Java-MVC-and-framework}

\section*{基本信息}
\sline
\begin{description}
\item[课程名称:] Java应用与开发
\item[授课教师:] 王晓东
\item[授课时间:] 第十四周
\item[参考教材:] 本课程参考教材及资料如下:
  \begin{itemize}
  \item 吕海东,张坤 编著,Java EE企业级应用开发实例教程,清华大学出版社,2010年8月
  \item 互联网资料
  \end{itemize}
\end{description}

\section*{教学目标}

\sline

\begin{enumerate}
\item 理解MVC设计模式的概念与特点,初步认识框架产生的基础。
\item 了解经典的MVC框架——Struts 2,学会使用Eclipse入手编写一个Struts 2
  Web应用。
\item 通过Struts 2的经典MVC框架设计和业务代码开发过程,进一步思考、理解
  框架。
\end{enumerate}  

\section*{授课方式}

\sline
\begin{description}
\item[理论课:] 多媒体教学、程序演示
\item[实验课:] 上机编程
\end{description}

\newpage
\section*{教学内容}
\sline

%%%%%%%%%%%%%%%%%%%%%%%%%%%%%%%%%%%%%%%%%%%%%%%%%%%%%%%%%%%%%%

\section{Java Web应用的开发演化}

\subsection{JSP方式}

JSP在HTML代码里写Java代码完成业务逻辑。

\begin{xmlCode}
<%
     String name = request.getParameter("name");
     String password = request.getParameter("password");

     UserHandler userHandler = new UserHandler();
     if(userHandler.authenticate(name, password)) {
%>
<p>Congratulations, login successfully. </p>
<%
      } else {
%>
<p>Sorry, login failed.</p>
<%
      }
%>
\end{xmlCode}

JSP方式仅有的一点优势包括:

\begin{enumerate}
\item 无需额外的配置文件,无需框架的帮助,即可完成逻辑。
\item 简单易上手。
\end{enumerate}

而JSP的劣势是相当明显的,包括:

\begin{enumerate}
\item Java代码由于混杂在一个HTML环境中而显得混乱不堪,可读性非常差。一
个JSP文件有时候会变成几十K,甚至上百K,经常难以定位逻辑代码的所在。
\item 编写代码时非常困惑,不知道代码到底应该写在哪里,也不知道别人是不
是已经曾经实现过类似的功能,到哪里去引用。
\item 突然之间,某个需求发生了变化。于是,每个人蒙头开始全程替换,还要
小心翼翼的,生怕把别人的逻辑改了。
\item 逻辑处理程序需要自己来维护生命周期,对于类似数据库事务、日志等众
多模块无法统一支持。
\end{enumerate}


在这个时候,如果有一种方式,它能够将页面上的那些Java代码抽取出来,让页
面上尽量少出现Java代码,该有多好。于是许多人开始使用servlet来处理那些业
务逻辑。

\subsection{Servlet方式}

\begin{javaCode}
  public class LoginServlet extends HttpServlet {
    
    @Override
    protected void doPost(HttpServletRequest req, HttpServletResponse resp) 
    throws ServletException, IOException {
      String message = null;
      RequestDispatcher dispatcher = req.getRequestDispatcher("/result.jsp");
      String name = req.getParameter("name");
      String password = req.getParameter("password");
      
      UserHandler userHandler = new UserHandler();
      if(userHandler.authenticate(name, password)) {
        message = "恭喜你,登录成功";
      } else {
        message = "对不起,登录失败";
      }
      
      req.setAttribute("message", message);
      dispatcher.forward(req, resp);
    }
  }
\end{javaCode}

\subsection{框架方式}

\begin{itemize}
\item 时代进一步发展,人们发现简单的JSP和Servlet已经很难满足人们懒惰的
  要求。于是,人们开始试图总结一些公用的Java类,来解决Web开发过程中碰到
  的问题。这时,横空出世了一个框架,叫做{\bf\Red Struts}。它非常先进地实现
  了{\hei\Red MVC模式},成为了广大程序员的福音。
\item 在一定程度上,Struts能够解决Web开发中的职责分配问题,使得显示与逻
  辑分开。
\item {\Blue\kai 不过开始的在很长一段时间里,学习使用Struts的程序员往
    往无法清晰的明白我们到底需要Web框架帮我们做什么,我们到底需要它完
    成点什么功能。}
\end{itemize}

\subsection{那么我们需要什么?}

在回顾写代码的历史之后,回头来看看,我们到底需要什么? 

无论是使用JSP,还是使用Struts1,或是Struts2,我们至少都需要一些必须的元素(如果没有这些元
素,或许我还真不知道这个程序会写成什么样子):

\begin{enumerate}
\item {\hei 数据} {在用户登录实例中就是name和password。他们共同
    构成了程序数据的核心载体。事实上,我们往往会有一个User类来封
    装name和password,这样会使得我们的程序更加OO。无论怎么说,数据会穿
    插在这个程序的各处,成为程序运行的核心。 }
\item {\hei 页面展示} {例如用户登录页面login.jsp。没有这个页面,
    一切的请求、验证和错误展示也无从谈起。在页面上,我们需要利用HTML,
    把我们需要展现的数据都呈现出来。同时我们也需要完成一定的页面逻辑,
    例如,错误展示,分支判断等。 }
\item {\hei 处理具体业务的场所} {不同阶段,处理具体业务的场所就
    不太一样。原来用JSP和Servlet,后来用Struts1或者Struts2的Action。}
\end{enumerate}

\subsection{MVC}

以上这些必须出现的元素,在不同的时代被赋予了不同的表现形式,有的受到时
代的束缚,其表现形式非常落后,有的已经不再使用。但是拨开这些外在的表现
形式,我们就可以发现,这就是我们已经熟悉的MVC。

\begin{itemize}
\item 数据 \ding{235} Model 
\item 页面展示 \ding{235} View 
\item 处理具体业务的场所 \ding{235} Control 
\end{itemize}

{\hei\Red 框架不重要。只要能够深刻理解MVC的概念,框架只是几个jar包而已。}

\begin{figure}[htb]
  \centering
  \includegraphics[width=0.8\textwidth]{images/Java-MVC-and-framework/MVC.pdf}
  \caption{MVC设计模式}
  \label{fig:mvc}
\end{figure}

MVC的特点主要包括:

\begin{enumerate}
\item 多个视图可以对应一个模型,可以减少代码的复制,在模型发生改变时,易于维护。
\item 模型返回的数据与显示逻辑分离。模型数据可以应用任何显示技术,例如,使用JSP、Velocity模板或者直接产生Excel。
\item 应用被分为三层,降低各层耦合,提高了可扩展性。
\item 控制层把不同模型和视图组合在一起,完成不同的请求,控制层包含了用户请求权限的概念。
\item MVC符合软件工程化管理的思想,不同层各司其职,有利于通过工程化和工具化产生管理程序代码。
\end{enumerate}



数据是动的,数据在View和Control层一旦运动起来,就会产生许多的问题:

\begin{itemize}
\item 数据从View层传递到Control层,如何使得一个个扁平的字符串,转化成一个个生龙活虎的Java对象。
\item 数据从View层传递到Control层,如何方便的进行数据格式和内容的校验? 
\item 数据从Control层传递到View层,一个个生龙活虎的Java对象,又如何在页面上以各种各样的形式展现出来 。
\item 如果你试图将数据请求从View层发送到Control层,你如何才能知道你要调用的究竟是哪个类,
  哪个方法?一个HTTP的请求,又如何与Control层的Java代码建立起关系来?
\end{itemize}

\subsection{框架}

框架是为了解决一个又一个在Web开发中所遇到的问题而诞生的。不同的框架,都
是为了解决不同的问题,但是对于程序员而言,他们仅仅是jar包而已。框架的优
缺点的评论,也完全取决于其对问题解决程度和解决方式的优雅性的评论。

所以,{\hei\Blue 千万不要为了学习框架而学习框架,而是要为了解决问题而学
  习框架,这才是一个程序员的正确学习之道 。}

\section{经典MVC框架 - Struts 2}

\subsection{为Web应用增加Struts 2支持}

下载安装Struts 2,登录http://struts.apache.org/download.cgi,下
载Struts 2的完整版(Full Distribution)。此处版本
为:struts-2.3.15.1-all.zip。

\begin{shCode}
  [18:10]xiaodong@Wang:~/Installed/struts-2.3.15.1[0]
  > ls
  ANTLR-LICENSE.txt       OGNL-LICENSE.txt        apps
  CLASSWORLDS-LICENSE.txt OVAL-LICENSE.txt        docs
  FREEMARKER-LICENSE.txt  SITEMESH-LICENSE.txt    lib
  LICENSE.txt             XPP3-LICENSE.txt        src
  NOTICE.txt              XSTREAM-LICENSE.txt
\end{shCode}

将Struts 2的lib目录下的{\Red\bf
  commons-fileupload-*.jar、commons-io-*.jar、freemarker-*.jar、
  javassist-*.jar、ognl-*.jar、struts2-core-*.jar、xwork-core-*.jar}必
备类库复制到{\bf\Red Web应用的WEB-INF/lib路径下}。

如果需要在DOS或Shell窗口下手动编译Strut 2相关程序,还需要
将struts2-core-*.jar和xwork-core-*.jar添加到系统的CLASSPATH环境变量。


编辑Web应用的web.xml配置文件,配置Strut 2的核心过滤器(Filter)。

\begin{xmlCode}\footnotesize
  <?xml version="1.0" encoding="GBK"?>
  <web-app xmlns:xsi="http://www.w3.org/2001/XMLSchema-instance" 
  xmlns="http://java.sun.com/xml/ns/javaee" 
  xmlns:web="http://java.sun.com/xml/ns/javaee/web-app_2_5.xsd" 
  xsi:schemaLocation="http://java.sun.com/xml/ns/javaee 
  http://java.sun.com/xml/ns/javaee/web-app_3_0.xsd" 
  id="WebApp_ID" version="3.0">

  <!-- 定义 Struts2 的核心 Filter -->
  <filter>
  <filter-name>struts2</filter-name>
  <filter-class>org.apache.struts2.dispatcher.ng.filter.
  StrutsPrepareAndExecuteFilter</filter-class>
  </filter>
  
  <!-- 让Struts2 的核心 Filter 拦截所有请求 -->
  <filter-mapping>
  <filter-name>struts2</filter-name>
  <url-pattern>/*</url-pattern>
  </filter-mapping>
  </web-app>   
\end{xmlCode}

使用Struts 2的功能需要一个struts.xml配置文件,默认放在Web应用的类加载路
径下(通常是WEB-INF/classes)。

经过上述步骤,我们可以在一个Web应用中使用Struts 2的基本功能。

\subsection{在Eclipse中使用Struts 2}

\subsubsection{创建并配置项目}

在Eclipse中创建动态Web项目:sample.struts2,按照上述步骤配置该项目。主
要包括:

\begin{itemize}
\item 添加依赖的Struts 2类库;
\item 在web.xml中加入并配置核心过滤器。
\end{itemize}

\subsubsection{增加登录处理}

以下示例为sample.struts2应用增加一个简单的登录处理流程,以简要介
绍Struts 2的开发步骤。

\tta{编写JSP页面}

\codeset{sample.struts2/WebContent/login.jsp}

\begin{xmlCode}
  <%@ page language="java" contentType="text/html; charset=UTF-8"
    pageEncoding="UTF-8"%>
  <%@taglib prefix="s" uri="/struts-tags"%>
  <!DOCTYPE html>
  <html>
  <head>
    <meta charset="UTF-8">
    <title><s:text name="loginPage" /></title>
  </head>
  <body>
    <h2>用户登录</h2>
    <s:form action="login">
      <s:textfield name="username" key="user"/>
      <s:password name="password" key="pass"/>
      <s:submit key="login"/>
    </s:form>
  </body>
  </html>  
\end{xmlCode}

上述login.jsp页面使用Struts 2标签库定义了一个表单和三个简单表单域。

\notice{注意}

几乎所有的MVC框架都会使用标签库,用以帮助开发者更加简单、更加规范的编写
视图组件(例如JSP页面)。

提供welcome.jsp页面和error.jsp页面,作为登录成功、登录失败后的提示页
面。

\ttc{sample.struts2/WebContent/welcome.jsp}

\begin{xmlCode}
<body>
  <s:text name="succTip">
    <s:param>\${sessionScope.user}</s:param>
  /s:text><br/>
</body>
\end{xmlCode}


\ttc{sample.struts2/WebContent/error.jsp}

\begin{xmlCode}
<body>
  <s:text name="failTip"/>
</body>
\end{xmlCode}
  
为了让Struts 2运行起来,还必须为Struts 2框架提供一个配置文件:{\bf\Red
  struts.xml}。

\ttc{sample.struts2/{\Red src}/struts.xml}

\begin{xmlCode}
  <?xml version="1.0" encoding="UTF-8"?>
  <!DOCTYPE struts PUBLIC
  "-//Apache Software Foundation//DTD Struts Configuration 2.1.7//EN"
  "http://struts.apache.org/dtds/struts-2.1.7.dtd">
  <!-- 指定 Struts 2 配置文件的根元素 -->
  <struts>
    <!-- 指定全局国际化资源文件 -->
    <constant name="struts.custom.i18n.resources" value="mess"/>
    <!-- 指定国际化编码所使用的字符集 -->	
    <constant name="struts.i18n.encoding" value="UTF-8"/>
    ...
  </struts>  
\end{xmlCode}

{\kai\Blue 注意:在Eclipse的管理下,当Eclipse生成、部署Web项目时,会自
  动将src路径下除了*.java外所有的文件都复制到Web应用的WEB-INF/classes路
  径下,所以struts.xml文件可以放在src目录。}

上述struts.xml文件中制定了国际化资源文件的base名为mess,所以需要为该应
用提供一个mess\_zh\_CN.properties文件。

\ttc{sample.struts2/src/mess\_XXX\_XXX.properties}

\begin{xmlCode}
  loginPage = 登录页面
  errorPage = 错误页面
  succPage = 成功页面
  failTip = 对不起,您不能登录!
  succTip = 欢迎, {0} ,您已经登录!
  user = 用户名
  pass = 密码
  login = 登录
\end{xmlCode}

注意,必须用native2ascii命令处理该国际化资源文件(Eclipse中自动完成)。

login.jsp页面中登录表单时指定改表单的action为login,所以必须定义一个Struts
2的Action,通常该继承ActionSupport基类。

\ttc{sample.struts2/src/ouc/java/app/action/LoginAction.java}

\begin{javaCode}
  public class LoginAction extends ActionSupport {
    // 定义封装请求参数的 username 和 password 属性
    private String username;
    private String password;
    public String getUsername() {
      return username;
    }
    public void setUsername(String username) {
      this.username = username;
    }
    public String getPassword() {
      return password;
    }
    public void setPassword(String password) {
      this.password = password;
    }
    // 定义处理用户请求的 execute 方法
    public String execute() throws Exception {
      if (getUsername().equals("admin") && getPassword().equals("admin")) {
        ActionContext.getContext().getSession().put("user", getUsername());
        return SUCCESS;
      } else {
        return ERROR;
      }
    }
  }
\end{javaCode}

\ttc{在struts.xml中配置action}

\begin{xmlCode}
... ...
<struts>
  <!-- 指定全局国际化资源文件 -->
  <constant name="struts.custom.i18n.resources" value="mess"/>
  <!-- 指定国际化编码所使用的字符集 -->	
  <constant name="struts.i18n.encoding" value="GBK"/>
  <!-- 所有的 Action 定义都应该放在 package 下 -->
  <package name="oucj2ee" extends="struts-default">
    <action name="login" class="ouc.java.app.action.LoginAction">
    <!-- 定义三个逻辑视图和物理资源之间的映射 -->		
      <result name="input">/login.jsp</result>
      <result name="error">/error.jsp</result>
      <result name="success">/welcome.jsp</result>
    </action>
  </package>
</struts>
\end{xmlCode}

{\Blue\kai 配置一个名称为login的Action,该Action配置三个result元素,用
  于指定逻辑视图与物理资源之间的映射。例如,当返回input逻辑视图名称时,
  系统跳转到/login.jsp页面。}

\section{Struts 2的开发步骤小结}

\subsection{在web.xml中配置核心过滤器}

在web.xml文件中增加如下配置片段:

\begin{xmlCode}
<!-- 定义 Struts 2 的核心过滤器 -->
<filter>
  <filter-name>struts2</filter-name>
  <filter-class>
    org.apache.struts2.dispatcher.ng.filter.StrutsPrepareAndExecuteFilter
  </filter-class>
</filter>
<!-- 让 Struts 2 的核心过滤器拦截所有请求 -->
<filter-mapping>
  <filter-name>struts2</filter-name>
  <url-pattern>/*</url-pattern>
</filter-mapping>  
\end{xmlCode}


\subsection{定义包含表单数据的JSP页面}

\begin{itemize}
\item 如果以{\bf\Red POST}方式提交表单数据,则定义包含表单数据的JSP页
  面。
\item 如果仅仅以{\bf\Blue GET}方式发送请求,则无需经过这一步。
\end{itemize}

\subsection{定义处理用户请求的Action类}

\begin{itemize}
\item Action是MVC中的C,即控制器。
\item 控制器Action负责调用Model里的方法来处理请求。
\item 在Struts 2中,MVC框架控制器实际上由两个部分组成:
  \begin{enumerate}\kai
  \item 拦截所有用户请求,处理请求的通用代码由核心控制器完成;
  \item 实际业务控制则有Action处理。
  \end{enumerate}
\end{itemize}

{\Blue\kai 注意:核心过滤器接收到用户请求后,通常会对用户请
  求进行简单预处理(例如解析、封装参数),然后通过反射来创建Action实例,
  并调用Action的指定方法(Struts 1通常是execute,Struts 2可以是任意方法)
  来处理用户请求。}

\subsection{在struts.xml中配置Action}

通常采用如下XML片段配置Action:

\begin{xmlCode}
  <action name="login" class="ouc.java.app.action.LoginAction">
  ... ...
  </action>
\end{xmlCode}

上述配置指定:用户请求的URL为login则使
用ouc.java.app.action.LoginAction来处理。

{\Blue\kai 注意:现在Struts 2的Convension插件借鉴Rails框架的
  优点,开始支持“约定优于配置”的思想,即采用约定的方式来规定用户请求地
  址和Action之间的对应关系。}

\subsection{配置处理结果和物理视图资源之间的对应关系}

\begin{itemize}
\item 当Action处理用户请求结束后,通常会返回一个处理结果(通常使用简单
  的字符串),我们可以认为该名称是{\hei\Red 逻辑视图名}。
\item 逻辑视图名需要和指定的物理视图资源关联才有价值,所以我们需要配置
  处理结果之间的对应关系。

  \begin{xmlCode}
    <action name="login" class="ouc.java.app.action.LoginAction">
      <!-- 定义3个逻辑视图和物理视图资源之间的映射 -->
      <result name="input">/login.jsp</result>
      <result name="error">/error.jsp</result>
      <result name="success">/success.jsp</result>
    </action>  
  \end{xmlCode}
\end{itemize}

\subsection{编写视图资源}

如果一个Action需要把一些数据传给视图资源,则可以借助{\Red\hei OGNL表达式}。

经过上述步骤后,我们基本完成了一个Struts 2处理流程的开发,即可以完整的执行一次HTTP请求/响
应过程。

\subsection{Struts 2流程}

\begin{figure}[htb]
  \centering
  \includegraphics[width=0.8\textwidth]{images/Java-MVC-and-framework/struts2.pdf}
  \caption{Strut2框架工作流程}
  \label{fig:strut2}
\end{figure}


\begin{itemize}\kai
\item StrutsPrepareAndExecuteFilter和XxxAction共同构成Strut 2的控制器,其中前者称为核心控
  制器,后者称为业务控制器。
\item 业务控制器并不与物理视图关联,这种做法提供了很好的解耦。
\item 在Struts 2的控制下,用户请求不再向JSP页面发送,而是由核心控制器来“调用”JSP页面来生
  成响应,此处调用不是直接调用,而是将请求forward到指定的JSP页面。
\end{itemize}

\section{课后习题}

\tta{简答题}

\begin{enumerate}
\item 什么是MVC设计模式?
\item MVC有哪些特点?
\item 总结Struts 2 Web应用开发的主要步骤。
\end{enumerate}

\tta{小编程}

\begin{enumerate}
\item 参考幻灯片步骤实践Struts 2 Web应用开发实例,初步了解Struts 2和基
  本的MVC框架开发模式。
\end{enumerate}
