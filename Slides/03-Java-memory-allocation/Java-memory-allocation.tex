%%%%%%%%%%%%%%%%%%%%%%%%%%%%%%%%%%%%%%%%%%%%%%%%%%%%%%%%%%%%%%%%%
\documentclass[hyperref={pdfpagelabels=false},compress,table]{beamer} % 在Mac下无法编译
% \documentclass[compress,table]{beamer} % 在Mac下使用
% package for font
\usepackage{fontspec}
\defaultfontfeatures{Mapping=tex-text}  %%如果没有它,会有一些 tex 特殊字符无法正常使用,比如连字符。
\usepackage{xunicode,xltxtra}
\usepackage[BoldFont,SlantFont,CJKnumber,CJKchecksingle]{xeCJK}  % \CJKnumber{12345}: 一万二千三百四十五
\usepackage{CJKfntef}  %%实现对汉字加点、下划线等。
\usepackage{pifont}  % \ding{}
% package for math
\usepackage{amsfonts}

% package for graphics
\usepackage[americaninductors,europeanresistors]{circuitikz}
\usepackage{tikz}
\usetikzlibrary{plotmarks}  % placements=positioning
\usepackage{graphicx}  % \includegraphics[]{}
\usepackage{subfigure}  %%图形或表格并排排列
% package for table
\usepackage{colortbl,dcolumn}  %% 彩色表格
\usepackage{multirow}
\usepackage{multicol}
\usepackage{booktabs}
% package for code
\usepackage{fancyvrb}
\usepackage{listings}

% \usepackage{animate}
% \usepackage{movie15}

%%%%%
% setting for beamer
\usetheme{default} % Madrid(常用), Copenhagen, AnnArbor, boxes(白色), Frankfurt,Berkeley
\useoutertheme[subsection=true]{miniframes} % 使用Berkeley时注释本行
\usecolortheme{sidebartab}
\usefonttheme{serif}  %%英文使用衬线字体
% \setbeamertemplate{background canvas}[vertical
% shading][bottom=white,top=structure.fg!7] %%背景色,上25%的蓝,过渡到下白。
\setbeamertemplate{theorems}[numbered]
\setbeamertemplate{navigation symbols}{}  %% 去掉页面下方默认的导航条
\setbeamercovered{transparent}  %设置 beamer 覆盖效果

% 设置标题title背景色
% \setbeamercolor{title}{fg=black, bg=lightgray!60!white}
\setbeamercolor{title}{fg=white, bg=black!70!white}

% 设置每页小LOGO
\pgfdeclareimage[width=1cm]{ouc}{figures/static/ouc.pdf}
\logo{\pgfuseimage{ouc}{\vspace{-20pt}}}

% setting for font
%%\setCJKmainfont{Adobe Kaiti Std}
\setCJKmainfont{SimSun} 
%% \setCJKmainfont{FangSong_GB2312} 
%%\setmainfont{Apple Garamond}  %%苹果字体没有SmallCaps
\setmainfont{Times New Roman} 
%FUNNY%\setCJKmainfont{DFPShaoNvW5-GB}  %%华康少女文字W5(P)
%FUNNY%\setCJKmainfont{FZJingLeiS-R-GB}  %%方正静蕾体
%FUNNY%\setmainfont{Purisa}
%\setsansfont[Mapping=tex-text]{Adobe Song Std}
     %如果装了Adobe Acrobat,可在font.conf中配置Adobe字体的路径以使用其中文字体。
     %也可直接使用系统中的中文字体如SimSun、SimHei、微软雅黑等。
     %原来beamer用的字体是sans family;注意Mapping的大小写,不能写错。
     %设置字体时也可以直接用字体名,以下三种方式等同:
     %\setromanfont[BoldFont={黑体}]{宋体}
     %\setromanfont[BoldFont={SimHei}]{SimSun}
     %\setromanfont[BoldFont={"[simhei.ttf]"}]{"[simsun.ttc]"}
% setting for graphics
\graphicspath{{figures/}}  %%图片路径
\renewcommand\figurename{图}

% setting for pdf
\hypersetup{% pdfpagemode=FullScreen,%
            pdfauthor={Xiaodong Wang},%
            pdftitle={Title},%
            CJKbookmarks=true,%
            bookmarksnumbered=true,%
            bookmarksopen=false,%
            plainpages=false,%
            colorlinks=true,%
            citecolor=green,%
            filecolor=magenta,%
            linkcolor=blue,%red(default)
            urlcolor=cyan}

% setting for fontspec
\XeTeXlinebreaklocale "zh"  %%表示用中文的断行
\XeTeXlinebreakskip = 0pt plus 1pt minus 0.1pt  %%多一点调整的空间
%%%%%

% font setting by xeCJK
\setCJKfamilyfont{NSimSun}{NSimSun}
\newcommand{\song}{\CJKfamily{NSimSun}}
%%%\setCJKfamilyfont{AdobeSongStd}{Adobe Song Std}
%%%\newcommand{\AdobeSong}{\CJKfamily{AdobeSongStd}}
\setCJKfamilyfont{FangSong}{FangSong_GB2312}
\newcommand{\fang}{\CJKfamily{FangSong}}
%%%\setCJKfamilyfont{AdobeFangsongStd}{Adobe Fangsong Std}
%%%\newcommand{\AdobeFang}{\CJKfamily{AdobeFangsongStd}}
\setCJKfamilyfont{SimHei}{SimHei}
\newcommand{\hei}{\CJKfamily{SimHei}}
%%%\setCJKfamilyfont{AdobeHeitiStd}{Adobe Heiti Std}
%%%\newcommand{\AdobeHei}{\CJKfamily{AdobeHeitiStd}}
\setCJKfamilyfont{KaiTi}{KaiTi}
\newcommand{\kai}{\CJKfamily{KaiTi}}
%%%\setCJKfamilyfont{AdobeKaitiStd}{Adobe Kaiti Std}
\newcommand{\AdobeKai}{\CJKfamily{AdobeKaitiStd}}
\setCJKfamilyfont{LiSu}{LiSu}
\newcommand{\li}{\CJKfamily{LiSu}}
\setCJKfamilyfont{YouYuan}{YouYuan}
\newcommand{\you}{\CJKfamily{YouYuan}}
\setCJKfamilyfont{FZJingLei}{FZJingLeiS-R-GB}
\newcommand{\jinglei}{\CJKfamily{FZJingLei}}
\setCJKfamilyfont{MSYH}{Microsoft YaHei}
\newcommand{\msyh}{\CJKfamily{MSYH}}

% 自定义颜色
\def\Red{\color{red}}
\def\Green{\color{green}}
\def\Blue{\color{blue}}
\def\Mage{\color{magenta}}
\def\Cyan{\color{cyan}}
\def\Brown{\color{brown}}
\def\White{\color{white}}
\def\Black{\color{black}}

\lstnewenvironment{javaCode}[1][]{% for Java
  \lstset{
    basicstyle=\tiny\ttfamily,%
    columns=flexible,%
    framexleftmargin=.7mm, %
    frame=shadowbox,%
    rulesepcolor=\color{cyan},%
    % frame=single,%
    backgroundcolor=\color{white},%
    xleftmargin=4\fboxsep,%
    xrightmargin=4\fboxsep,%
    numbers=left,numberstyle=\tiny,%
    numberblanklines=false,numbersep=7pt,%
    language=Java, %
    }\lstset{#1}}{}

\lstnewenvironment{shCode}[1][]{% for Java
  \lstset{
    basicstyle=\scriptsize\ttfamily,%
    columns=flexible,%
    framexleftmargin=.7mm, %
    frame=shadowbox,%
    rulesepcolor=\color{brown},%
    % frame=single,%
    backgroundcolor=\color{white},%
    xleftmargin=4\fboxsep,%
    xrightmargin=4\fboxsep,%
    numbers=left,numberstyle=\tiny,%
    numberblanklines=false,numbersep=7pt,%
    language=sh, %
    }\lstset{#1}}{}

\newcommand\ask[1]{\vskip 4bp \tikz \node[rectangle,rounded corners,minimum size=6mm,
  fill=white,]{\Cyan \includegraphics[height=1.5cm]{question} \Large \msyh #1};}

\newcommand\wxd[1]{\vskip 4bp \tikz \node[rectangle,minimum size=6mm,
  fill=blue!60!white,]{\White \ding{118} \msyh #1};}

\newcommand\xyy[1]{\vskip 2bp \tikz \node[rectangle,minimum size=3mm,
  fill=black!80!white,]{\White \msyh\scriptsize #1};}

\newcommand\cxf[1]{\vskip 4bp \tikz \node[rectangle,rounded corners,minimum size=6mm,
  fill=orange!60!white,]{\White \ding{42} \msyh #1};}

\newcommand\samp[1]{\vskip 2bp \tikz \node[rectangle,minimum size=3mm,
  fill=white!100!white,]{\Mage\msyh \small CODE \ding{231} \Black #1};\vskip -8bp}

\newcommand\zhyfly[1]{\tikz \node[rectangle,rounded corners,minimum size=6mm,ball color=red!25!blue,text=white,]{#1};}

\setbeamerfont{frametitle}{series=\msyh} % 修改Beamer标题字体

\makeatletter
\newcommand{\Extend}[5]{\ext@arrow 0099{\arrowfill@#1#2#3}{#4}{#5}}
\makeatother
%%%%%%%%%%%%%%%%%%%%%%%%%%%%%%%%%%%%%%%%%%%%%%%%%%%%%%%%%%%%%%%%%

%%%%%%%%%%%%%%%%%%%%%%%%%%%%%%%%%%%%%%%%%%%%%%%%
% \titlepage
\title[KevinW@OUC]{\hei {\huge Java 应用程序设计}\\  
 Java内存分配机制}
\author[KevinW@OUC]{KevinW@OUC\\
  \href{mailto:wangxiaodong@ouc.edu.cn}{\footnotesize wangxiaodong@ouc.edu.cn}}
\institute[中国海洋大学]{\small 中国海洋大学}
\date{\today}
\titlegraphic{\vspace{-6em}\includegraphics[height=6cm]{static/ouc.pdf}\vspace{-6em}}
%%%%%
\begin{document}
%% Delete this, if you do not want the table of contents to pop up at
%% the beginning of each subsection:
\AtBeginSection[]{                              % 在每个Section前都会加入的Frame
  \frame<handout:0>{
    \frametitle{\textbf{\hei 接下来…}}
    \tableofcontents[currentsection]
  }
}  %

\AtBeginSubsection[]                            % 在每个子段落之前
{
  \frame<handout:0>                             % handout:0 表示只在手稿中出现
  {
    \frametitle{\textit{\hei 接下来…}}\small
    \tableofcontents[current,currentsubsection] % 显示在目录中加亮的当前章节
  }
}
 \frame{\titlepage}

%%%%%%%%%%%%%%%%%%%%%%%%%%%%%%%%%%%%%%%%%%%%%%%%
\section*{大纲}
\frame{\frametitle{大纲} \tableofcontents }

\section{Java内存模型}

\begin{frame}[fragile] % [fragile]参数使得能够插入代码
\frametitle{以JVM的视角}

Java程序运行在JVM(Java Virtual Machine,Java虚拟机)上,可以把JVM理解成Java程序和操作系
统之间的桥梁,JVM实现了Java的平台无关性。JVM是内存分配原理的前提。
\end{frame}

\begin{frame}[fragile] % [fragile]参数使得能够插入代码
\frametitle{JVM内存模型}

\begin{figure}
\centering
\includegraphics[width=0.8\textwidth]{jmm.png}
\end{figure}

{\kai
从大方面讲,JVM的内存模型分为两块:{\hei 永久区内存(Permanent space)和堆内存
  (Heap space)}。栈内存(stack space)一般不归在JVM内存模型中,因为栈内存属于线程级别,
每个线程都有个独立的栈内存空间。}
\end{frame}

\begin{frame}[fragile] % [fragile]参数使得能够插入代码
\frametitle{Java程序运行过程会涉及的内存区域}
\begin{itemize}
\item Permanent space里存放加载的Class类级对象如class本身、method、field等。
\item Heap space主要存放对象和数组。\\
\only<2>{\kai Heap space由Old Generation和New Generation组成,Old Generation存放生命周期长久的实例对象,而新的对象实例一般放在New Generation。New Generation还可以再分为Eden区和Survivor区,新的对象实例总是首先放在Eden区,Survivor区作为Eden区和Old区的缓冲,可以向Old区转移活动的对象实例。}
\end{itemize}
\end{frame}

\begin{frame}[fragile] % [fragile]参数使得能够插入代码
\frametitle{JVM堆内存空间申请流图}

\begin{figure}
\centering
\includegraphics[width=0.7\textwidth]{jmmflow.jpg}
\end{figure}
\end{frame}

\begin{frame}[fragile] % [fragile]参数使得能够插入代码
\frametitle{JVM内存溢出和参数调优\footnote{关于参数调优需勘误。}}

\begin{itemize}
\item 常见的OOM(Out Of Memory)内存溢出异常,就是堆内存空间不足以存放新对象实例时导致。
\item 永久区内存溢出相对少见,一般是由于需要加载海量的Class数据,超过了非堆内存的容量导致。
  通常出现在Web应用刚刚启动时,因此Web应用推荐使用预加载机制,方便在部署时就发现并解决该
  问题。
 \item 栈内存也会溢出,但是更加少见。
\end{itemize}

\begin{description}\scriptsize
\item[堆内存优化] 调整JVM启动参数-Xms -Xmx -XX:newSize -XX:MaxNewSize,如调整初始堆内存和
  最大对内存 -Xms256M -Xmx512M。 或者调整初始New Generation的初始内存和最大内
  存 -XX:newSize=128M -XX:MaxNewSize=128M。
 \item[永久区内存优化] 调整PermSize参数   如  -XX:PermSize=256M -XX:MaxPermSize=512M。
 \item[栈内存优化] 调整每个线程的栈内存容量  如  -Xss2048K。
\end{description}
\end{frame}

\begin{frame}[fragile] % [fragile]参数使得能够插入代码
\frametitle{Java程序运行过程会涉及的内存区域}
\begin{description}\kai\small
\item[寄存器] JVM内部虚拟寄存器,存取速度非常快,程序不可控制。
\item[栈] 保存局部变量的值,包括:用来保存基本数据类型的值;保存类的实例,即堆区对象的引
  用(指针),也可以用来保存加载方法时的帧。(Stack)
\item[堆] 用来存放动态产生的数据,如new出来的对象\footnote{注意创建出来的对象只包含属于
    各自的成员变量,并不包括成员方法。因为同一个类的对象拥有各自的成员变量,存储在各自的
    堆内存中,但是他们共享该类的方法,并不是每创建一个对象就把成员方法复制一次。}。(Heap)
\item[常量池] JVM为每个已加载的类型维护一个常量池,常量池就是这个类型用到的常量的一个有序
  集合。包括直接常量(基本类型,String)和对其他类型、方法、字段的符号引用。池中的数据和
  数组一样通过索引访问,常量池在Java程序的动态链接中起了核心作用。(Perm)
\item[代码段] 存放从硬盘上读取的源程序代码。(Perm)
\item[数据段] 存放static定义的静态成员。{\Red (Perm)} 
\end{description}
\end{frame}

\section{Java程序内存运行分析}

\begin{frame}[fragile] % [fragile]参数使得能够插入代码
\frametitle{预备知识}
\begin{enumerate}
\item 一个Java文件,只要有main入口方法,即可认为这是一个Java程序,可以单独编译运行。
\item 无论是普通类型的变量还是引用类型的变量(俗称实例),都可以作为局部变量,他们都可以
  出现在栈中。只不过普通类型的变量在栈中直接保存它所对应的值,而引用类型的变量保存的是一
  个指向堆区的指针。通过这个指针,就可以找到这个实例在堆区对应的对象。因此,{\hei 普通类型变量
  只在栈区占用一块内存,而引用类型变量要在栈区和堆区各占一块内存}。
\end{enumerate}
\end{frame}

\begin{frame}[fragile] % [fragile]参数使得能够插入代码
\frametitle{所用讲解程序实例}
\samp{Test.java}
\begin{javaCode}
public class Test {
  public static void main(String[] args) {
    Test test = new Test();
    int data = 9;
    BirthDate d1 = new BirthDate(22, 12, 1982);
    BirthDate d2 = new BirthDate(10, 10, 1958);
    test.m1(data);
    test.m2(d1);
    test.m3(d2);
  }

  public void m1(int i) {
    i = 1234;
  }
  public void m2(BirthDate b) {
    b = new BirthDate(15, 6, 2010);
  }
  public void m3(BirthDate b) {
    b.setDay(18);
  }
}
\end{javaCode}
\end{frame}

\begin{frame}[fragile] % [fragile]参数使得能够插入代码
\frametitle{程序调用过程(一)}

\begin{columns}
\column{0.4\textwidth} 
\begin{figure}
\centering
\includegraphics[width=0.98\textwidth]{fig01.pdf}
\end{figure}

\column{0.6\textwidth}  
\begin{javaCode}\small
public class Test {
  public static void main(String[] args) {
    Test test = new Test(); //1
    int data = 9; //2
    BirthDate d1 = new BirthDate(22, 12, 1982); //3
    BirthDate d2 = new BirthDate(10, 10, 1958); //4
    test.m1(date);
    test.m2(d1);
    test.m3(d2);
  }

  public void m1(int i) {
    i = 1234;
  }
  public void m2(BirthDate b) {
    b = new BirthDate(15, 6, 2010);
  }
  public void m3(BirthDate b) {
    b.setDay(18);
  }
}
\end{javaCode}
\end{columns}
\end{frame}

\begin{frame}[fragile] % [fragile]参数使得能够插入代码
\frametitle{程序调用过程(一)}
\begin{itemize}
\item JVM自动寻找main方法,执行第一句代码,创建一个Test类的实例,在栈中分配一块内存,存放
  一个指向堆区对象的指针110925。
\item 创建一个int型的变量date,由于是基本类型,直接在栈中存放date对应的值9。
\item 创建两个BirthDate类的实例d1、d2,在栈中分别存放了对应的指针指向各自的对象。它们在实
  例化时调用了有参数的构造方法,因此对象中有自定义初始值。
\end{itemize}
\end{frame}

\begin{frame}[fragile] % [fragile]参数使得能够插入代码
\frametitle{程序调用过程(二)}

\begin{columns}
\column{0.4\textwidth} 
\begin{figure}
\centering
\includegraphics[width=0.98\textwidth]{fig02.pdf}
\end{figure}

\column{0.6\textwidth}  
\begin{javaCode}\small
public class Test {
  public static void main(String[] args) {
    Test test = new Test(); 
    int data = 9; 
    BirthDate d1 = new BirthDate(22, 12, 1982); 
    BirthDate d2 = new BirthDate(10, 10, 1958); 
    test.m1(date); //5
    test.m2(d1);
    test.m3(d2);
  }

  public void m1(int i) {
    i = 1234;
  }
  public void m2(BirthDate b) {
    b = new BirthDate(15, 6, 2010);
  }
  public void m3(BirthDate b) {
    b.setDay(18);
  }
}
\end{javaCode}
\end{columns}
\end{frame}

\begin{frame}[fragile] % [fragile]参数使得能够插入代码
\frametitle{程序调用过程(二)}
\begin{itemize}
\item 调用test对象的m1方法,以date为参数。JVM读取这段代码时,检测到i是局部变量,则会把i放
  在栈中,并且把date的值赋给i。
\end{itemize}
\end{frame}

\begin{frame}[fragile] % [fragile]参数使得能够插入代码
\frametitle{程序调用过程(三)}

\begin{columns}
\column{0.4\textwidth} 
\begin{figure}
\centering
\includegraphics[width=0.98\textwidth]{fig03.pdf}
\end{figure}

\column{0.6\textwidth}  
\begin{javaCode}\small
public class Test {
  public static void main(String[] args) {
    Test test = new Test(); 
    int data = 9; 
    BirthDate d1 = new BirthDate(22, 12, 1982); 
    BirthDate d2 = new BirthDate(10, 10, 1958); 
    test.m1(date); 
    test.m2(d1);
    test.m3(d2);
  }

  public void m1(int i) {
    i = 1234; //6
  }
  public void m2(BirthDate b) {
    b = new BirthDate(15, 6, 2010);
  }
  public void m3(BirthDate b) {
    b.setDay(18);
  }
}
\end{javaCode}
\end{columns}
\end{frame}

\begin{frame}[fragile] % [fragile]参数使得能够插入代码
\frametitle{程序调用过程(三)}
\begin{itemize}
\item 把1234赋值给i。
\end{itemize}
\end{frame}

\begin{frame}[fragile] % [fragile]参数使得能够插入代码
\frametitle{程序调用过程(四)}

\begin{columns}
\column{0.4\textwidth} 
\begin{figure}
\centering
\includegraphics[width=0.98\textwidth]{fig04.pdf}
\end{figure}

\column{0.6\textwidth}  
\begin{javaCode}\small
public class Test {
  public static void main(String[] args) {
    Test test = new Test(); 
    int data = 9; 
    BirthDate d1 = new BirthDate(22, 12, 1982); 
    BirthDate d2 = new BirthDate(10, 10, 1958); 
    test.m1(date); 
    test.m2(d1); 
    test.m3(d2);
  }

  public void m1(int i) {
    i = 1234; 
  }
  public void m2(BirthDate b) {
    b = new BirthDate(15, 6, 2010);
  }
  public void m3(BirthDate b) {
    b.setDay(18);
  }
}
\end{javaCode}
\end{columns}
\end{frame}

\begin{frame}[fragile] % [fragile]参数使得能够插入代码
\frametitle{程序调用过程(四)}
\begin{itemize}
\item m1方法执行完毕,立即释放局部变量i所占用的栈空间。。
\end{itemize}
\end{frame}

\begin{frame}[fragile] % [fragile]参数使得能够插入代码
\frametitle{程序调用过程(五)}

\begin{columns}
\column{0.4\textwidth} 
\begin{figure}
\centering
\includegraphics[width=0.98\textwidth]{fig05.pdf}
\end{figure}

\column{0.6\textwidth}  
\begin{javaCode}\small
public class Test {
  public static void main(String[] args) {
    Test test = new Test(); 
    int data = 9; 
    BirthDate d1 = new BirthDate(22, 12, 1982); 
    BirthDate d2 = new BirthDate(10, 10, 1958); 
    test.m1(date); 
    test.m2(d1); //7
    test.m3(d2);
  }

  public void m1(int i) {
    i = 1234; 
  }
  public void m2(BirthDate b) { 
    b = new BirthDate(15, 6, 2010);
  }
  public void m3(BirthDate b) {
    b.setDay(18);
  }
}
\end{javaCode}
\end{columns}
\end{frame}

\begin{frame}[fragile] % [fragile]参数使得能够插入代码
\frametitle{程序调用过程(五)}
\begin{itemize}
\item 调用test对象的m2方法,以实例d1为参数。JVM检测到m2方法中的b参数为局部变量,立即加入
  到栈中,由于是引用类型的变量,所以b中保存的是d1中的指针,此时b和d1指向同一个堆中的对象。
  在b和d1之间传递是指针。
\end{itemize}
\end{frame}

\begin{frame}[fragile] % [fragile]参数使得能够插入代码
\frametitle{程序调用过程(六)}

\begin{columns}
\column{0.4\textwidth} 
\begin{figure}
\centering
\includegraphics[width=0.98\textwidth]{fig06.pdf}
\end{figure}

\column{0.6\textwidth}  
\begin{javaCode}\small
public class Test {
  public static void main(String[] args) {
    Test test = new Test(); 
    int data = 9; 
    BirthDate d1 = new BirthDate(22, 12, 1982); 
    BirthDate d2 = new BirthDate(10, 10, 1958); 
    test.m1(date); 
    test.m2(d1); 
    test.m3(d2);
  }

  public void m1(int i) {
    i = 1234; 
  }
  public void m2(BirthDate b) { 
    b = new BirthDate(15, 6, 2010); //8
  }
  public void m3(BirthDate b) {
    b.setDay(18);
  }
}
\end{javaCode}
\end{columns}
\end{frame}

\begin{frame}[fragile] % [fragile]参数使得能够插入代码
\frametitle{程序调用过程(六)}
\begin{itemize}
\item m2方法中又实例化了一个BirthDate对象,并且赋给b。在内部执行过程是:在堆区new了一个对
  象,并且把该对象的指针保存在栈中b对应空间,此时实例b不再指向实例d1所指向的对象,但是实
  例d1所指向的对象并无变化,未对d1造成任何影响。
\end{itemize}
\end{frame}

\begin{frame}[fragile] % [fragile]参数使得能够插入代码
\frametitle{程序调用过程(七)}
\begin{itemize}
\item m2方法执行完毕,立即释放局部引用变量b所占的栈空间,注意只是释放了栈空间,堆空间要等待自动回收。
\end{itemize}
\end{frame}

\begin{frame}[fragile] % [fragile]参数使得能够插入代码
\frametitle{程序调用过程(八)}
\begin{itemize}
\item 调用test实例的m3方法,以实例d2为参数。JVM会在栈中为局部引用变量b分配空间,并且
  把d2中的指针存放在b中,此时d2和b指向同一个对象。再调用实例b的setDay方法,其实就是调
  用d2指向的对象的setDay方法。
\item 调用实例b的setDay方法会影响d2,因为二者指向的是同一个对象。
\item m3方法执行完毕,立即释放局部引用变量b。
\end{itemize}
\end{frame}

\begin{frame}[fragile] % [fragile]参数使得能够插入代码
\frametitle{小结}
\begin{itemize}\kai
\item 基本类型和引用类型,二者作为局部变量都存放在栈中。基本类型直接在栈中保存值,引用类
  型只保存一个指向堆区的指针,真正的对象存放在堆中。作为参数时基本类型就直接传值,引用类
  型传指针。
\item 注意什么是对象。
\begin{javaCode}
  Class a = new Class();    
\end{javaCode}
此时a是指向对象的指针,而不能说a是对象。指针存储在栈中,对象存储在堆中,操作实例实际上是
通过指针间接操作对象。多个指针可以指向同一个对象。
\end{itemize}
\end{frame}

\begin{frame}[fragile] % [fragile]参数使得能够插入代码
\frametitle{小结(续)}
\begin{itemize}\kai
\item 栈中的数据和堆中的数据销毁并不是同步的。方法一旦结束,栈中的局部变量立即销毁,但是
  堆中对象不一定销毁。因为可能有其他变量也指向了这个对象,直到栈中没有变量指向堆中的对象
  时,它才销毁;而且还不是马上销毁,要等垃圾回收扫描时才可以被销毁。
\item 以上的栈、堆、代码段、数据段等都是相对于应用程序而言的。每一个应用程序都对应唯一的
  一个JVM实例,每一个JVM实例都有自己的内存区域,互不影响。并且这些内存区域是所有线程共享
  的。
\item 类的成员变量在不同对象中各不相同,都有自己的存储空间(成员变量在堆中的对象中)。而
  类的方法却是该类的所有对象共享的。
\end{itemize}
\end{frame}

\section{常量池技术}

\begin{frame}[fragile] % [fragile]参数使得能够插入代码
\frametitle{基本类型和基本类型的包装类}

\wxd{基本类型}\\
byte、short、char、int、long、boolean

\wxd{基本类型的包装类}\\
Byte、Short、Character、Integer、Long、Boolean

\wxd{二者的区别}\\
基本类型体现在程序中是普通变量,基本类型的包装类是类,体现在程序中是引用变量。因此二者在内存中的存储位置不同:基本类型存储在栈中,而基本类型包装类存储在堆中。

{\hei 上述包装类均实现了常量池技术,另外两种浮点数类型的包装类则没有实现。String类型也实现了常量池技术。}
\end{frame}

\begin{frame}[fragile] % [fragile]参数使得能够插入代码
\frametitle{代码实例}
\begin{javaCode}
public class Test {
  public static void main(String[] args) {    
    objPoolTest();
  }
  
  public static void objPoolTest() {
    int i = 40;
    int i0 = 40;
    Integer i1 = 40;
    Integer i2 = 40;
    Integer i3 = 0;
    Integer i4 = new Integer(40);
    Integer i5 = new Integer(40);
    Integer i6 = new Integer(0);
    Double d1 = 1.0;
    Double d2 = 1.0;
        
    System.out.println("i=i0\t" + (i == i0));
    System.out.println("i1=i2\t" + (i1 == i2));
    System.out.println("i1=i2+i3\t" + (i1 == i2 + i3));
    System.out.println("i4=i5\t" + (i4 == i5));
    System.out.println("i4=i5+i6\t" + (i4 == i5 + i6));    
    System.out.println("d1=d2\t" + (d1==d2)); 
    System.out.println();        
  }
}  
\end{javaCode}
\end{frame}

\begin{frame}[fragile] % [fragile]参数使得能够插入代码
\frametitle{Output}
\begin{stdoutCode}
i=i0    true
i1=i2   true
i1=i2+i3    true
i4=i5   false
i4=i5+i6    true
d1=d2   false  
\end{stdoutCode}
\end{frame}

\begin{frame}[fragile] % [fragile]参数使得能够插入代码
\frametitle{Output analysis 1}\kai
\begin{itemize}
\item i和i0均是普通类型(int)的变量,所以数据直接存储在栈中,而栈有一个很重要的特性:{\Blue 栈
  中的数据可以共享}。当我们定义了int i = 40;,再定义int i0 = 40;,这时候会自动检查栈中是否
  有40这个数据,如果有,i0会直接指向i的40,不会再添加一个新的40。
\item i1和i2均是引用类型,在栈中存储指针,因为Integer是包装类,实现了常量池技术,因
  此i1、i2的40均是从常量池中获取的,均指向同一个地址,因此i1 = i2。
\item 很明显这是一个加法运算,Java的数学运算都是在栈中进行的,Java会自动对i1、i2进行拆箱
  操作转化成整型,因此i1在数值上等于i2 + i3。
\end{itemize}
\end{frame}

\begin{frame}[fragile] % [fragile]参数使得能够插入代码
\frametitle{Output analysis 2}
\begin{itemize}\kai
\item i4和i5均是引用类型,在栈中存储指针,因为Integer是包装类。但是由于他们各自都是new出
  来的,因此不再从常量池寻找数据,而是从堆中各自new一个对象,然后各自保存指向对象的指针,
  所以i4和i5不相等,因为他们所存指针不同,所指向对象不同。
\item 也是一个加法运算,同理3。
\item d1和d2均是引用类型,在栈中存储指针,因为Double是包装类。但{\Red Double包装类没有实现常量
  池技术},因此Double d1 = 1.0;相当于Double d1 = new Double(1.0);,是在堆中new一个对象,d2同理。
  因此d1和d2存放的指针值不同,指向的对象不同,所以不相等。
\end{itemize}
\end{frame}

\begin{frame}[fragile] % [fragile]参数使得能够插入代码
\frametitle{Output analysis 3}
\begin{itemize}\kai
\item 以上提到的几种基本类型包装类均实现了常量池技术,但他们{\Red 维护的常量仅仅是
    从-128至127这个范围内},如果常量值超过这个范围,就会从堆中创建对象,不再从常量池中取。
  比如,把上边例子改成Integer i1 = 400; Integer i2 = 400;,很明显超过了127,无法从常量池
  中获取常量,就要从堆中new新的Integer对象,这时i1和i2就不相等。
\item String类型也实现了常量池技术,但是稍有不同。String型是先检测常量池中有没有对应字符
  串,如果有则取出来;如果没有则把当前的添加进去。
\begin{javaCode}
// s1, s2 分别位于堆中不同空间
String s1 = new String("hello");
String s2 = new String("hello");
System.out.println(s1 == s2); // 输出 false
// s3, s4 位于池中同一空间
String s3 = "hello";
String s4 ="hello";
System.out.println(s3 == s4);// 输出 true
\end{javaCode}
\end{itemize}
\end{frame}

\section{Java内存建议}
\begin{frame}[fragile] % [fragile]参数使得能够插入代码
\frametitle{Java人为的内存管理是必要的} 

Java需要内存管理。虽然JVM已经代替开发者完成了对内存的管理,但是硬件本身
的资源是有限的,如果Java的开发人员不注意内存的使用依然会造成较高的内存
消耗,导致性能的降低。

\end{frame}

\begin{frame}[fragile] % [fragile]参数使得能够插入代码
\frametitle{Java Garbage Collection} 

{\hei JVM决定对象是否是垃圾对象,并进行回收。}

垃圾内存并不是用完了马上就被释放,所以会产生内存释放不及时的现象,从而降低内存的使用效率。
而当程序庞大的时候,这种现象更为明显,并且垃圾回收(GC)工作需要消耗资源,同样会产生内存
浪费。

\wxd{JVM中的对象生命周期}\\
对象的生命周期一般分为7个阶段:\ding{182}创建阶段、\ding{183}应用阶段、\ding{184}不可视阶段、\ding{185}不可到达阶段、\ding{186}可收集阶段、\ding{187}终结阶段、\ding{188}释放阶段。
\end{frame}

\begin{frame}[fragile] % [fragile]参数使得能够插入代码
\frametitle{创建阶段}
\wxd{减少无谓的对象引用创建}
\samp{Test 1}
\begin{javaCode}
for( int i=0; i<10000; i++) {
  Object obj=new Object(); 
}
\end{javaCode}
\samp{Test 2}
\begin{javaCode}
Object obj=null; 
for( int i=0; i<10000; i++) {
  obj=new Object(); 
}
\end{javaCode}
\xyy{分析}\\
{\small\kai Test 2的性能要比Test 1性能要好,内存使用率要高。两段程序每次执行for循环都要创
  建一个Object的临时对象,JVM的垃圾回收不会马上销毁但这些临时对象。相对于Test 1,Test 2则
  只在内存中保存一份对象的引用,而不必创建大量新临时变量,从而降低了内存的使用。}
\end{frame}

\begin{frame}[fragile] % [fragile]参数使得能够插入代码
\frametitle{创建阶段}
\wxd{不要对同一对象初始化多次}
\samp{Test}
\begin{javaCode}
public class A { 
  private Hashtable table = new Hashtable(); 
  public A() { 
    table = new Hashtable(); // 应该去掉,因为table已被初始化
  } 
\end{javaCode}

\xyy{分析}\\
{\small\kai 上述代码new了两个Hashtable,但是却只使用了一个,另外一个则没有被引用而被忽略
  掉,浪费了内存。并且由于进行了两次new操作,也影响了代码的执行速度。另外,不要提前创建对
  象,尽量在需要的时候创建对象。}
\end{frame}

\begin{frame}[fragile] % [fragile]参数使得能够插入代码
\frametitle{其他阶段}
\begin{description}
\item[应用] 即该对象至少有一个引用在维护它。
\item[不可视] 即超出该变量的作用域。\\{\kai 因为JVM GC并不是马上进行回收,而是要判断对象
    是否被其他引用维护。所以,如果我们在使用完一个对象以后对其进行obj =
    null或者obj.doSomething()操作,将其标记为空,则帮助JVM及时发现这个垃圾对象。}
\item[不可到达] 即在JVM中找不到对该对象的直接或者间接的引用。
\item[可收集,终结,释放] 垃圾回收器发现该对象不可到达,finalize方法已经被执
  行,或者对象空间已被重用的时候。
\end{description}
\end{frame}



\begin{frame}[fragile] % [fragile]参数使得能够插入代码
  \frametitle{Java的finalize()方法}

Java所有类都继承自Object类,而finalize()是Object类的一个函数,该函数在Java中类似于C++的析
构函数(仅仅是类似)。一般来说可以通过重载finalize()的形式来释放类中对象。

\begin{javaCode}
public class A { 
  public Object a; 

  public A() { 
    a = new Object() ;
  } 
  
  protected void finalize() throws java.lang.Throwable { 
    a = null; // 标记为空,释放对象 
    super.finalize(); // 递归调用超类中的 finalize 方法
  }
} 
\end{javaCode}

什么时候finalize()被调用由JVM来决定。{\hei\Blue 尽量少用finalize()函数,finalize()函数是Java提供给程
序员一个释放对象或资源的机会。但它会加大GC的工作量,因此尽量少采用finalize方式回收资源。}
\end{frame}

\begin{frame}[fragile] % [fragile]参数使得能够插入代码
  \frametitle{Java的finalize()方法}
  \begin{itemize}
  \item 一般的,纯Java编写的Class不需要重写finalize()方法,因为Object已
    经实现了一个默认的,除非我们要实现特殊的功能。
  \item 用Java以外的代码编写的Class(比如JNI、C++的new方法分配的内存),
    垃圾回收器并不能对这些部分进行正确的回收,这就需要我们覆盖默认的方
    法来实现对这部分内存的正确释放和回收。
  \end{itemize}
\end{frame}

%%%%%%%%%%%%%%%%%%%%%%%%%%
\begin{frame}[fragile] % [fragile]参数使得能够插入代码
\frametitle{课后作业}
\begin{enumerate}
\item 搜索关于Java常量池技术的相关文档和资源,并进行总结。
\item 搜索关于Java中比较操作“==”和“equals()”的相关文档,并进行总结,加
  深对Java内存模型的理解。
\end{enumerate}
\end{frame}

% TKS %%%%%%%%%%%%%%%%%%%%%%%%%%%%%%%%%%%%%%%%%%%%%%
\begin{frame}
\centering
{\Huge \textcolor{blue}{THE END}} \\
\vspace{5mm}
{\Large wxd2870@163.com} \\
\end{frame}
%%%%%%%%%%%%%%%%%%%%%%%%%%%%%%%%%%%%%%%%%%%%%%%%%%
\end{document}
