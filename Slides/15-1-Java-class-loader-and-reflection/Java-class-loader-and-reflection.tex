%%%%%%%%%%%%%%%%%%%%%%%%%%%%%%%%%%%%%%%%%%%%%%%%%%%%%%%%%%%%%%%%%%%%%%%%%%%%%%%
\input{../SLIDE-MAIN/slide-main}
%%%%%%%%%%%%%%%%%%%%%%%%%%%%%%%%%%%%%%%%%%%%%%%%%%%%%%%%%%%%%%%%%%%%%%%%%%%%%%%
% \titlepage
\title[KevinW@OUC]{\hei {\Large Java应用与开发}\\  
  类加载和反射}
\author[王晓东]{王晓东\\
  \href{mailto:wangxiaodong@ouc.edu.cn}{\footnotesize wangxiaodong@ouc.edu.cn}}
\institute[中国海洋大学]{\small 中国海洋大学}
\date{\today}
\titlegraphic{\vspace{-6em}\includegraphics[height=6cm]{static/ouc.pdf}\vspace{-6em}}
%%%%%
\begin{document}
%% Delete this, if you do not want the table of contents to pop up at
%% the beginning of each subsection:
\AtBeginSection[]{                              % 在每个Section前都会加入的Frame
  \frame<handout:0>{
    \frametitle{\textbf{\hei 接下来…}}
    \tableofcontents[currentsection]
  }
}  %

\AtBeginSubsection[]                            % 在每个子段落之前
{
  \frame<handout:0>                             % handout:0 表示只在手稿中出现
  {
    \frametitle{\textit{\hei 接下来…}}\small
    \tableofcontents[current,currentsubsection] % 显示在目录中加亮的当前章节
  }
}
\frame{\titlepage}
%%%%%%%%%%%%%%%%%%%%%%%%%%%%%%%%%%%%%%%%%%%%%%%%
\begin{frame}
\frametitle{学习目标}
\begin{enumerate}
\item 理解什么是反射机制,通过常见场景认识反射的作用。
\item 掌握类的加载、连接和初始化概念。
\item 理解类加载器及类加载机制。
\item 掌握使用反射生成并操作对象的方法。
\end{enumerate}  
\end{frame}
 
\section*{大纲}
\frame{\frametitle{大纲} \tableofcontents }


\section{反射}

\begin{frame}[fragile] % [fragile]参数使得能够插入代码
  \frametitle{反射机制}

  \begin{itemize}
  \item 程序运行时,允许改变程序结构或变量类型,这种语言称为动态语言。从
    这个观点看,Perl、Python、Ruby是动态语言,C++、Java、C\#不是动态语
    言。
  \item 但是Java有着一个非常突出的动态相关机制:{\Red 反射
      (Reflection)},可以于运行时加载、探知、使用编译期间完全未知的类。
    换句话说,Java程序可以加载一个运行时才得知名称的类,获悉其完整构造
    (但不包括methods定义),并生成其对象实体、或对其fields设值、或唤起
    调用其methods。
  \end{itemize}
\end{frame}

\begin{frame}[fragile] % [fragile]参数使得能够插入代码
  \frametitle{反射机制}

  \begin{itemize}
  \item 反射机制是Java语言在运行时所拥有的一项自观能力。通过这种能力可以
    彻底的了解自身的情况为下一步的动作做准备。
  \item 反射机制是在运行状态(而不是编译状态)时:

    \begin{itemize}\kai
    \item 判断任意一个对象所属的类;
    \item 构造任意一个类的对象;
    \item 判断任意一个类所具有的成员变量和方法(通过反射甚至可以调用private方法);
    \item 调用任意一个对象的方法。
    \end{itemize}

  \end{itemize}
\end{frame}


\begin{frame}[fragile] % [fragile]参数使得能够插入代码
  \frametitle{反射的主要用途\ding{182}}

  \wxd{一个常见的场景}

  当我们在使用IDE(如Eclipse)时,当我们输入一个对象或类并想调用它的属
  性或方法,一按点号,编译器就会自动列出它的属性或方法,这里就会用到反
  射。

\end{frame}

\begin{frame}[fragile] % [fragile]参数使得能够插入代码
  \frametitle{反射的主要用途\ding{183}}

  \wxd{反射最重要的用途就是开发各种通用框架}

  很多框架(比如Spring、Strut)都是基于配置化的,比如通过XML文件配
  置JavaBean和Action。为了保证框架的通用性,需要根据配置
  文件加载不同的对象或类,调用不同的方法,这个时候就必须用到反射——{\hei\Red 运行
    时动态加载需要加载的对象}。

  \xyy{一个例子} {\kai Struts2框架开发中会在struts.xml里配置Action}

  \begin{xmlCode}
    <action name="login" class="ouc.j2ee.action.LoginAction" method="execute">
      <result name="success">index.jsp</result>
      <result name="error">login.jsp</result>
    </action>
  \end{xmlCode}
  
  {\small\Blue XML配置文件与Action实现建立了映射关系。用户请求login.action会
  被StrutsPrepareAndExecuteFilter拦截并解
  析struts.xml文件,检索其中name为login的Action,并根据class属性创
  建LoginAction的实例,并用invoke方法来调用execute方法。}

  {\Red 这个过程是基于Java反射框架完成的}。

\end{frame}

\begin{frame}[fragile] % [fragile]参数使得能够插入代码
  \frametitle{反射的主要用途\ding{184}}

  \wxd{依赖注入}

  有两个组件A和B,A依赖于B。

  \begin{javaCode}
    public class A {
      public void importantMethod() {
        B b = ...; // get an instance of B
        b.usefulMethod();
        ...
      }
    }
  \end{javaCode}

  {\bf\Red 我们需要获得B的实例的引用。如果B是接口且有多个实现该怎么做?}

\end{frame}

\begin{frame}[fragile] % [fragile]参数使得能够插入代码
  \frametitle{反射的主要用途3}

  \wxd{依赖注入}

  接管对象的创建工作,并将该对象的引用注入需要该对象的组件。例如,我们
  使用Spring\footnote{对Spring依赖注入框架感兴趣的请自行搜索学习。}框架
  将对象B注入到对象A中,A需要进行如下修改,加入setB()方法:

  \begin{javaCode}
    public class A {
      private B b;
      public void importantMethod() {
        b.usefulMethod();
        ...
      }

      public void setB(B b) {
        this.b = b;
      }
    }   

  \end{javaCode}


  \begin{xmlCode}
    <bean id="b" class="ouc.j2ee.sample.B" />
    <bean id="a" class="ouc.j2ee.sample.A">
      <property name="b" ref="b"></property>
    </bean>
  \end{xmlCode}
\end{frame}

\section{类的加载、连接和初始化}

\begin{frame}[fragile] % [fragile]参数使得能够插入代码
\frametitle{JVM和类}

\tta{JVM进程终止的条件}

同一个JVM的所有线程、所有变量都处在同一个进程里,它们都使用该JVM进程的
内存区。当系统出现以下几种情况时,JVM进程将被终止:

\begin{itemize}\kai
\item 程序运行至正常结束。
\item 程序运行到使用System.exit()或Runtime.getRuntime().exit()代码结束程序。
\item 程序执行过程中遇到未捕获的异常或者错误而结束。
\item 程序所在的平台强制结束了JVM进程。
\end{itemize}

\pause\tta{JVM初始化类的步骤}

{\hei 当程序主动使用某个类时,如果该类还未被加载到内存中,JVM会通
  过{\Red 加载 \ding{233} 连接 \ding{233} 初始化}三个步骤对该类进行初始
  化。}
\end{frame}

\begin{frame}[fragile] % [fragile]参数使得能够插入代码
\frametitle{类的加载}

类的加载是指将类的class文件读入内存,并为之创建一个java.lang.Class对象。

{\Red\kai(类是某一类对象的抽象,都是java.lang.Class的实例)}

\tta{类加载的要点}

\begin{itemize}\kai
\item JVM提供类加载器(系统类加载器)来完成对类的加载。
\item 此外,开发者可以通过继承ClassLoader基类来创建自己的类加载器。
\item 可以从本地文件系统、jar包和网络方式加载类的class文件。
\item 类加载器通常无须等到“首次使用”该类时才加载此类,Java虚拟机允许系统预先加载某些类。
\end{itemize}
\end{frame}

\begin{frame}[fragile] % [fragile]参数使得能够插入代码
\frametitle{类的连接}

类被加载生成对应的Class对象后,进入连接阶段,负责把类的二进制数据合并到JRE中。

\begin{enumerate}
\item 验证:用于检验被加载的类是否有正确的内部结构,并和其它类协调一致。
\item 准备:负责为类的静态属性分配内存,并设置默认初始值。
\item 解析:将类的二进制数据中的符号引用替换成直接引用。
\end{enumerate}
\end{frame}

\begin{frame}[fragile] % [fragile]参数使得能够插入代码
\frametitle{类的初始化}

主要负责对静态属性进行初始化:
\begin{itemize}
\item 声明静态属性时指定初始值。
\item 使用静态初始化块为静态属性指定初始值。
\end{itemize}

\begin{javaCode}
public class Test {
  static int a = 5;
  static int b;
  static int c;
  static {
    b = 6;
  }
  ... ...
}  
\end{javaCode}
初始化上述代码后:a=5 b=6 c=0。

{\kai\Red 注意:程序主动使用一个类时,系统会保证该类以及所有父类(包括
  直接父类和间接父类)都会被初始化,所有JVM最先初始化的总
  是java.lang.Object类。}
\end{frame}

\begin{frame}[fragile] % [fragile]参数使得能够插入代码
\frametitle{类的初始化时机}

当Java程序首次通过下面6种方式使用某个类或接口时,系统会初始化该类或接口:
\begin{itemize}\kai
\item 创建类的实例,包括通过new操作符、通过反射、通过反序列化的方式。
\item 调用某个类的静态方法。
\item 访问某个类或接口的静态属性,或为该静态属性赋值。
\item 使用反射方式来强制创建某个类或接口对应的java.lang.Class对象。例如,Class.forName("Person")。
\item 初始化某个类的子类,其父类都会被初始化。
\item 直接使用java.exe运行某个主类时。
\end{itemize}


\end{frame}

\begin{frame}[fragile] % [fragile]参数使得能够插入代码
\frametitle{类的初始化时机}

当使用ClassLoader类的loadClass()方法来加载某个类时,该方法只是加载该类,并不会执行该类的初始化。

\begin{javaCode}
ClassLoader cl = ClassLoader.getSystemClassLoader();
cl.loadClass("Tester");  
\end{javaCode}

当使用Class的forName()静态方法时导致强制初始化该类。
\begin{javaCode}
Class.forName("Tester");  
\end{javaCode}
\end{frame}

\section{类加载器}

\begin{frame}[fragile] % [fragile]参数使得能够插入代码
\frametitle{类加载器概述}

类加载器负责加载类,在内存中生成java.lang.Class的实例。一个载入JVM的类
也有一个唯一的标识。

\notice{注意}

{\kai 在Java中代码中,一个类用其全限定名作为标识。在JVM中,一个类用其全限定名
  和其类加载器作为唯一标识。

  例如,包pg中的Person类,被类加载器ClassLoader的实例k1负责加载,则
  该Person类对应的Class对象在JVM中表示为(Person, pg, k1)。

  类加载器不同,即使加载同一个类,所加载的类的实例也是完全不同、互不兼
  容的。}

\end{frame}

\begin{frame}[fragile] % [fragile]参数使得能够插入代码
\frametitle{类加载器层次结构}

JVM启动时,会形成由三个类加载器组成的初始类加载器层次结构:

\begin{itemize}\kai
\item Bootstrap ClassLoader \ding{235} 根类加载器
\item Extension ClassLoader \ding{235} 扩展类加载器
\item System ClassLoader \ding{235} 系统类加载器
\end{itemize}
\end{frame}

\begin{frame}[fragile] % [fragile]参数使得能够插入代码
  \frametitle{类加载器层次结构}
  
  \begin{description}
  \item[Bootstrap] ClassLoader{\Red 根类加载器}负责加载Java的核心类,它非常特
    殊,并不是java.lang.ClassLoader的子类,而是由JVM自身实现的。
  \item[Extension] ClassLoader{\Blue 扩展类加载器},负责加载JRE的扩展目录
    (JAVA\_HOME/jre/lib/ext或者java.ext.dirs系统属性指定的目录中
    的JAR类的包。
  \item[System] ClassLoader{\Mage 系统(应用)类加载器},负责在JVM启动时,加载
    来自java中的-classpath选项或java.class.path系统属性,或CLASSPATH环
    境变量所指定JAR包和类路径。程序可以通过ClassLoader的静态方
    法getSystemClassLoader()获得该类加载器。没有特别指定,用户自定义的
    类加载器都以该类加载器作为父加载器。
\end{description}


\end{frame}

\begin{frame}[fragile] % [fragile]参数使得能够插入代码
\frametitle{类加载机制}

\wxd{关于类加载机制的几点说明}

\begin{description}
\item[全盘负责] 当一个类加载器复负责加载某个Class时,该Class所依赖和引
  用的其它Class也将由该类加载器负责载入,除非显式使用另一个类加载器载
  入。
\item[父类委托] 先让父类加载器试图加载该Class,只有父类加载器无法加载该
  类时才尝试从自己的类路径中加载该类。
\item[缓存机制] 类加载器先从缓存中搜索Class,只有当缓存中不存在
  该Class对象时,系统才会重新读取该类对应的二进制数据。
\end{description}

\codeset{sample.classloader.ClassLoaderSample.java}
\end{frame}

\begin{frame}[fragile] % [fragile]参数使得能够插入代码
\frametitle{类加载机制}

\tta{代码执行结果}

{\footnotesize
\begin{verbatim}
系统类加载器为 sun.misc.Launcher$AppClassLoader@2a139a55
系统类加载器的加载路径为 ...
扩展类加载器为 sun.misc.Launcher$ExtClassLoader@7852e922
扩展类加载器的加载路径为 ...
扩展类加载器的Parent为 null
\end{verbatim}}

\tta{分析说明}
  
\begin{itemize}
\item {\Red\kai 扩展类加载器的getParent()方法返回null,并不是
    根类加载器。这是因为根类加载器没有继承自ClassLoader抽象类,所以返回
    空。但实际上,根类加载器确实是扩展类加载器的父类加载器。}
\item 系统类加载器是AppClassLoader的实例,扩展类加载器
  是ExtClassLoader的实例;实际上,这两个类都是URLClassLoader的实例。
\end{itemize}
\end{frame}

\begin{frame}[fragile] % [fragile]参数使得能够插入代码
\frametitle{类加载器加载Class的大致步骤}
\begin{enumerate}\kai
\item 检查此Class是否载入过(即缓存中是否存在),如果有则直接进入第8步,否则进入第2步。
\item 如果父加载器不存在(如果父加载器不存在,要么parent一定是根加载器,要么本身就是根加载器),则跳到第4步。如果父加载器存在,则接着执行第3步。
\item 请求父加载器加载目标类,如果成果则进入第8步,否则执行第5步。
\item 请求使用根加载器载入目标类,如果成功则进入第8步,否则跳到第7步。
\item 从与此ClassLoader相关的类路径中寻找Class文件,如果找到则执行第6步,否则跳到第7步。
\item 从文件中载入Class,成功载入后则跳到第8步。
\item 抛出ClassNotFoundException异常。
\item 返回Class。
\end{enumerate}
\end{frame}

\begin{frame}[fragile] % [fragile]参数使得能够插入代码
\frametitle{URLClassLoader类}
Java为ClassLoader提供了一个URLClassLoader实现类,该类是系统类加载器和扩展类加载器的父类。
该类既可以从本地文件系统获取二进制文件来加载类,也可以从远程主机获取二进制文件来加载类。

URLClassLoader提供了如下两个构造器:

\begin{itemize}
\item URLClassLoader(URL[] urls) 使用默认的父类加载器创建一个ClassLoader对象,该对象将
  从urls所指定的系列路径中查询并加载类。
\item URLClassLoader(URL[] urls, ClassLoader parent) 使用指定的父类加载器创建一
  个ClassLoader对象。
\end{itemize}
一旦得到了URLClassLoader对象后,就可以调用该对象的loadClass方法加载指定类。
\end{frame}

\begin{frame}[fragile] % [fragile]参数使得能够插入代码
\frametitle{从文件系统中加载MySQL驱动的示例}

\begin{javaCode}
URL[] urls = {new URL("file:mysql-connector-java-***-bin.jar")};
URLClassLoader myClassLoader = new URLClassLoader(urls);
Driver driver = (Driver) myClassLoader.loadClass("com.mysql.jdbc.Driver").newInstance();
... ...
Connection conn = driver.connect("jdbc:mysql://localhost:3306/mysql", props);
\end{javaCode}

\begin{itemize}\kai
\item 该类加载器的加载路径为当前路径下
  的mysql-connector-java-***-bin.jar文件。这里file:前缀表明从本地文件系
  统加载,也可以以http:或ftp:为前缀,表示通过网络加载。
\item 使用ClassLoader的loadClass加载指定类,并调用Class对象
  的newInstance()方法创建一个该类的实例。
\end{itemize}
\end{frame}




\section{使用反射生成并操作对象}

\begin{frame}[fragile] % [fragile]参数使得能够插入代码
\frametitle{获得Class对象的三种方式}

\begin{enumerate}
\item 使用Class类的forName()静态方法。该方法需要传入字符串参数,为某个类的全限定名(包含完整的包名)。
\item 调用某个类的class属性获取该类的Class对象,例如Person.class将返回Person类对应的Class对象。
\item 调用某个对象的getClass()方法,该方法是java.lang.Object类的一个方法,该方法会返回该对象所属类对应的Class对象。
\end{enumerate}

第二种方式更建议使用,具有两个优势:
\begin{itemize}\kai
\item 代码更安全,程序在编译阶段就可以检查需要访问的Class对象是否存在。
\item 程序性能更高,因为这种方式无需调用方法,所以具有更好的性能。
\end{itemize}
\end{frame}

\begin{frame}[fragile] % [fragile]参数使得能够插入代码
\frametitle{使用反射生成并操作对象}

Class对象可以获得该类里的成分,包括:

\begin{description}
\item[方法] 由Method对象表示,可以通过该对象执行对应的方法。
\item[构造器] 由Constructor对象表示,可以通过该对象来调用对应的构造器创建对象。
\item[Field] 由Field对象表示,可以通过该对象直接访问并修改对象属性值。
\end{description}

{\Red\small 上述三个类都定义在java.lang.reflect包下,并实现
了java.lang.reflect.Member接口。}

\end{frame}

\begin{frame}[fragile] % [fragile]参数使得能够插入代码
\frametitle{创建对象}

使用反射来生成对象的两种方式:

\begin{itemize}
\item 使用Class对象的newInstance()方法来创建该Class对象对应的类实例,此种方法要求
  该Class对象的对应类有默认构造器,而执行newInstance()方法时实际利用默认构造器来创建该类
  的实例。
\item 先使用Class对象获取指定的Constructor对象,再调用Constructor对象的newInstance()方法
  创建该Class对象对应类的实例。通过这种方式可以选择使用某个类的指定构造器来创建实例。
\end{itemize}
\end{frame}

\begin{frame}[fragile] % [fragile]参数使得能够插入代码
\frametitle{创建对象示例}

很多Java EE框架中都需要根据配置文件信息来创建Java对象。从配置文件中读取的只是某个类的字符串名,程序就需要根据该字符串创建对应的实例,就必须使用{\hei\Red 反射}。

\wxd{简单对象池的示例}

\begin{javaCode}
import java.io.FileInputStream;
import java.io.IOException;
import java.util.HashMap;
import java.util.Map;
import java.util.Properties;

public class ObjectPoolFactory {
  //定义对象池
  private Map<String, Object> objectPool = new HashMap<String, Object>();
  
  private Object createObject(String ClazzName)
  throws ClassNotFoundException, InstantiationException,
  IllegalAccessException {
    //根据字符串来获取对应的 Class 对象
    Class<?> clazz = Class.forName(ClazzName);
    return clazz.newInstance();
  }
\end{javaCode}
\end{frame}


\begin{frame}[fragile] % [fragile]参数使得能够插入代码
\frametitle{创建对象示例(续)}
\begin{javaCode}
  //根据指定文件来初始化对象池
  public void initPool(String fileName) throws ClassNotFoundException,
  InstantiationException, IllegalAccessException {
    FileInputStream fis = null;
    try {
      fis = new FileInputStream(fileName);
      Properties props = new Properties();
      props.load(fis);
      for (String name : props.stringPropertyNames()) {
        objectPool.put(name, createObject(props.getProperty(name)));
      }
    } catch (IOException ex) {
      System.out.println("读取" + fileName + "异常");
    } finally {
      try {
        if (fis != null) {
          fis.close();
        }
      } catch (IOException ex) {
        ex.printStackTrace();
      }
    }
  }
\end{javaCode}
\end{frame}

\begin{frame}[fragile] % [fragile]参数使得能够插入代码
\frametitle{创建对象示例(续)}

\begin{javaCode}
  public Object getObject(String name) {
    return objectPool.get(name);
  }
  
  public static void main(String[] args) throws ClassNotFoundException,
  InstantiationException, IllegalAccessException {
    ObjectPoolFactory pf = new ObjectPoolFactory();
    pf.initPool("obj.txt");
    System.out.println(pf.getObject("a"));
  }
}  
\end{javaCode}

File: obj.txt

\begin{xmlCode}
a=java.util.Date
b=javax.swing.JFrame  
\end{xmlCode}

\end{frame}

\begin{frame}[fragile] % [fragile]参数使得能够插入代码
\frametitle{使用指定的构造器创建对象}

需要利用Constructor对象,每个Constructor对应一个构造器,步骤如下:
\begin{enumerate}
\item 获取该类的Class对象。
\item 利用Class对象的getConstructor()方法来获得指定构造器。
\item 调用Constructor()的newInstance()方法创建对象。
\end{enumerate}
\end{frame}

\begin{frame}[fragile] % [fragile]参数使得能够插入代码
\frametitle{使用指定的构造器创建对象}

\begin{javaCode}
import java.lang.reflect.Constructor;
import java.lang.reflect.InvocationTargetException;

public class CreateJFrame {
  public static void main(String[] args) throws ClassNotFoundException,
  IllegalArgumentException, InstantiationException,
  IllegalAccessException, InvocationTargetException,
  SecurityException, NoSuchMethodException {
    // 获取 JFrame 对应的 Class 对象
    Class<?> jframeClazz = Class.forName("javax.swing.JFrame");
    // 获取 JFrame 中带一个字符串参数的构造器
    Constructor ctor = jframeClazz.getConstructor(String.class);
    Object obj = ctor.newInstance("测试窗口");
    System.out.println(obj);
  }
}  
\end{javaCode}
注意:
\begin{itemize}\scriptsize
\item 如果要唯一的确定某类中的构造器,只要指定构造器的{\Red 形参列表}即可。
\item 调用Constructor对象的newInstance()方法时通常需要传入参数,实际上等于调用它对应的构造器。
\item 只有当程序需要动态创建某个类的对象时才会考虑使用反射;对于已知类的情形,通常没有必要通过反射创建对象(降低性能)。
\end{itemize}
\end{frame}

\begin{frame}[fragile] % [fragile]参数使得能够插入代码
\frametitle{调用方法}

获得某个类对应的Class对象后,可以该对象的如下方法执行方法调用:
\begin{itemize}
\item getMethods()方法:获取全部方法,返回值为Method对象数组;
\item getMethod()方法:获取指定方法,返回Method对象。
\end{itemize}
获取Method对象后,程序可以该对象的invoke()方法调用对应方法:
\begin{javaCode}
Object invoke(Object obj, Object ... args)
\end{javaCode}
其中,obj是主调,args是执行该方法时传入该方法的参数。
\end{frame}

\begin{frame}[fragile] % [fragile]参数使得能够插入代码
\frametitle{调用方法}
\begin{javaCode}
Class<?> targetClass = target.getClass();
Method mtd = targetClass.getMethod(mtdName, String.class);
mtd.invoke(target, props.getProperty(name));
\end{javaCode}

当通过Method对象的invoke方法调用对应方法时,Java会要求程序必须有调用该
方法的权限。{\Red 如果程序需要调用某个对象的private方法,可以先调
  用Method对象的如下方法}:

\begin{itemize}\kai
\item setAccessible(boolean flag) 将Method对象的accessible标志设置为指
  示的布尔值。值为true则指示该Method在使用时应该取消Java语言访问权限检
  查,值为flase则指示该Method在使用时应该实施Java语言访问权限检查。
\end{itemize}
\end{frame}

\begin{frame}[fragile] % [fragile]参数使得能够插入代码
\frametitle{访问属性值}

通过Class对象的getFields()或getField()方法可以获取该类所包含的全部Field(属性)或指定Field。
\begin{itemize}\kai
\item getXxx(Object obj) 获取obj对象该Field的属性值,此处Xxx对应8个基本类型,如果该属性的
  类型为引用类型则取消get后面的Xxx。
\item setXxx(Object obj, Xxx val) 将obj对象的该Field设置成val值。此处Xxx对应8个基本类型,
  如果该属性的类型为引用类型则取消get后面的Xxx。
\end{itemize}
\end{frame}

\begin{frame}[fragile] % [fragile]参数使得能够插入代码
\frametitle{访问属性值}
\begin{javaCode}
import java.lang.reflect.Field;
class Person{
  private String name;
  private int age;
  public String toString(){
    return "Person [ name: " + name + ", age: " + age + " ]"; 
  }
}

public class FieldTest {
  public static void main(String[] args) throws Exception {
    Person p = new Person();
    Class<Person> personClazz = Person.class;
    Field nameField = personClazz.getDeclaredField("name");
    nameField.setAccessible(true);
    nameField.set(p, "Kevin W");
    Field ageField = personClazz.getDeclaredField("age");
    ageField.setAccessible(true);
    ageField.setInt(p, 30);
    System.out.println(p);
  }
}
\end{javaCode}

上述代码使用getDeclaredField()方法获取名为name的Field,而不是使用getField()方法,因
为getField()方法只能获取public的Field,而getDeclaredField()则可以获取所有Field。
\end{frame}

\begin{frame}[fragile] % [fragile]参数使得能够插入代码
\frametitle{操作数组}

java.lang.reflect包提供Array类,代表所有数组,程序可以通过Array类来动态的创建数组,操作数组元素。

\begin{itemize}
\item static Object newInstance(Class<?> componentType, int... length) 创建一个具有指定
  元素类型、指定维度的新数组。
\item static xxx getXxx(Object array, int index) 返回array数组中第index个元素,其中xxx是
  各种基本数据类型,如果数组元素为引用类型,则方法去掉Xxx,为get(Object array, int
  index)。
\item static void setXxx(Object array, int index, xxx val) 将array数组中第index元素的值设
  为val。其中xxx是各种基本数据类型,如果数组元素为引用类型,则方法去掉Xxx,为set(Object
  array, int index, Object val)。
\end{itemize}
\end{frame}

\begin{frame}[fragile] % [fragile]参数使得能够插入代码
\frametitle{操作数组}
\begin{javaCode}
//创建一个元素类型为 String ,长度为 10 的数组
Object arr = Array.newInstance(String.class, 10);
//为数组中 index 为 5 的元素赋值
Array.set(arr, 5, "Java应用与开发");
//取出 arr 数组中 index 为 5 的元素的值
Object book = Array.get(arr, 5);

//创建一个元素类型为 String 的三维数组
Object arr = Array.newInstance(String.class, 3, 4, 10);
//获取 arr 数组中 index 为 2 的元素,是二维数组
Object arrObj = Array.get(arr1, 2);
//获取 arrObj 数组中 index 为 3 的元素,应该是一维数组
Object Arr.get(arrObj, 3);
//将 arr 强制转换为三维数组
String[][][] cast = (String[][][]) arr;
\end{javaCode}
\end{frame}

%%\begin{frame}[fragile] % [fragile]参数使得能够插入代码
%%\frametitle{使用反射生成JDK动态代理}
%%\end{frame}

\section{本节习题}

\begin{frame}
  \frametitle{本节习题}

  \tta{简答题}
  
  \begin{enumerate}
  \item JVM类初始化机制是怎样的?
  \item Java类加载器有哪些特点?
  \item 类加载器的层次结构是怎样的?各层分别用于加载哪些类?
  \item 什么是反射?
  \end{enumerate}
\end{frame}
  
\begin{frame}
  \frametitle{本节习题}
  
  \tta{小编程}

  \begin{enumerate}
  \item 尝试编程应用反射机制动态操作对象。
  \end{enumerate}
\end{frame}
% TKS %%%%%%%%%%%%%%%%%%%%%%%%%%%%%%%%%%%%%%%%%%%%%%
\begin{frame}
\centering
{\Huge \textcolor{blue}{THE END}} \\
\vspace{5mm}
{\Large wangxiaodong@ouc.edu.cn} \\
\end{frame}
%%%%%%%%%%%%%%%%%%%%%%%%%%%%%%%%%%%%%%%%%%%%%%%%%%
\end{document}
